\documentclass{article} % For LaTeX2e
\usepackage{nips13submit_e,times}
\usepackage{hyperref}
\usepackage{amsmath}
\usepackage{algorithm}
\usepackage[noend]{algpseudocode}
\usepackage{url}
%\documentstyle[nips13submit_09,times,art10]{article} % For LaTeX 2.09

\makeatletter
\def\BState{\State\hskip-\ALG@thistlm}
\makeatother

\title{Word2Bits - Quantized Word Vectors}


\author{
Maximilian Lam\\
\texttt{maxlam@stanford.edu} \\
}

\newcommand{\fix}{\marginpar{FIX}}
\newcommand{\new}{\marginpar{NEW}}

\nipsfinalcopy % Uncomment for camera-ready version

\begin{document}


\maketitle

\begin{abstract}
Word vectors require significant amounts of memory and storage, posing
issues to resource limited devices like mobile phones and GPUs. We
show that high quality quantized word vectors using 1-2 bits per
parameter can be learned by training Word2Vec with a quantization
function. We furthermore show that training with the quantization
function acts as a regularizer. We train word vectors on full english
Wikipedia (2017) and evaluate them on standard word similarity and
analogy tasks and on question answering (SQuAD). Our quantized word
vectors not only take 10-16x less space than full precision (32 bit)
word vectors but also outperform them on word similarity tasks and
question answering.
\end{abstract}

\section{Introduction}
Word vectors are extensively used in deep learning models for natural
language processing. Each word vector is typically represented as a
300-500 dimensional vector, with each parameter being 32 bits. As
there are millions of words, word vectors may take up to 3-6 GB of
memory/storage -- a massive amount relative to other portions of a
deep learning model[1]. These requirements pose issues to
memory/storage limited devices like mobile phones and GPUs.

Furthermore, word vectors are often re-trained on application specific
data for better performance in application specific domains[1]. This
motivates directly learning high quality compact word representations
rather than adding an extra layer of compression on top of pretrained
word vectors which may be computational expensive and degrade
accuracy.

Recent trends indicate that deep learning models can reach a high
accuracy even while training in the presence of significant noise and
perturbation[1]. It has furthermore been shown that high quality
quantized deep learning models for image classification can be learned
at the expense of more training epochs[1]. Inspired by these trends we
ask: can we learn high quality word vectors such that each parameter
is only one of two values, or one of four values (quantizing to 1 and
2 bits respectively)?

To that end we propose learning quantized word vectors by introducing
a quantization function into the Word2Vec loss formulation -- we call
our simple change Word2Bits. While introducing a quantization function
into a loss function is not new, to the best of our knowledge it is
the first time it has been applied to learning compact word representations.

In this report we show that
\begin{enumerate}
\item[$\bullet$]

  It is possible to train high quality quantized word vectors which
  take 10x-16x less storage/memory than full precision word
  vectors. Experiments on both intrinsic and extrinsic tasks show that
  our learned word vectors perform comparably or even better on many tasks.

\item[$\bullet$]

  Standard Word2Vec may be prone to overfitting; the quantization
  function acts as a regularizer against it.

\end{enumerate}

\section{Related Work}
Word vectors are continuous representations of words and are used by
most deep learning NLP models. Word2Vec, introduced by Mikolov's
groundbreaking papers[1], is a unsupervised neural network algorithm for
learning word vectors from textual data. Since then, other
groundbreaking algorithms (Glove, FastText) [1] have been proposed to
learn word vectors using other properties of textual data. As of 2018
the most widely used word vectors are Glove, Word2Vec and
FastText. This work focuses on how to learn memory/storage efficient
word vectors through quantized training -- specifically our approach
extends Word2Vec to output high quality quantized word vectors.

Learning compact word vectors is related to learning compressed neural
networks. Finding compact representations of neural
networks date back to the 90's and include techniques like network
pruning[1], knowledge distillation[1], deep compression[1] and
quantization[1]. More recently, algorithmic and hardware advances have
allowed training deep models using low precision floating-point and
arithmetic operations[1] -- this is also referred to as
quantization. To distinguish between quantized training with low
precision arithmetic/floats from quantized training with full
precision arithmetic/floats but constrained values we term the first
physical quantization and the latter virtual quantization.

Our technical approach follows that of neural network quantization for
image classification[1], which does virtual quantization by
introducing a sign function (a 1 bit quantization function) into the
training loss function. The actual technique of backpropagating through a
discrete function (the quantization function) has been thoroughly
explored by Hinton[1] and Bengio[1].

Application wise, various techniques exist to compress word
embeddings[1]. These approaches involve taking pre-trained word
vectors and compressing them using dimensionality reduction[1],
pruning[1], or more complicated approaches like deep compositional
coding[1]. Such techniques add an extra layer of computation to
compress pre-trained embeddings and also may degrade word vector
performance[1].

To the best of our knowledge, current traditional methods of obtaining
compact word vectors involve adding an extra layer of computation to
compress pretrained word vectors[1] (as described previously). This
may incur computational costs which may be expensive in context of
retraining word vectors for application specific purposes and may
degrade word vector performance[1]. Our proposed approach of directly
learning quantized word vectors from textual data may amend these
issues and is an alternative method of obtaining compact high quality
word vectors. Note that these traditional compression methods may
still be applied on the learned quantized word vectors.

\section{Word2Bits - Quantized Word Embeddings}
\subsection{Background}
Our approach utilizes the Word2Vec formulation of learning word vectors. There
are two Word2Vec algorithms: Skip Gram Negative Sampling (SGNS) and
Continuous Bag of Words (CBOW)[1] -- our virtual quantization
technique utilizes CBOW with negative sampling. The CBOW negative
sampling loss function minimizes

$$
J(u_o, \hat{v}_c) = -log(\sigma(u_o^T\hat{v}_c)) - \sum_{i=1}^{k} log(\sigma(-u_i^T\hat{v}_c))
$$

where

$$
u_o = \mbox{ vector of center word with corpus position } o
$$
$$
\hat{v}_c = \frac{1}{2w-1}\sum_{-w+o \leq i \leq w+o,i \neq o} v_i \mbox{  where } v_i \mbox{ is vector for context word, } w \mbox{ is window size, a hyperparameter}
$$
$$
k = \mbox{ number of negative samples, a hyperparameter}
$$

Intuitively, minimizing this loss function optimizes vectors of
words that occur in similar contexts to be ``closer'' to each other,
and pushes vectors whose contexts are different ``away''. Specifically CBOW with negative sampling tries to predict the
center word from context words.

Technically, to optimize this loss function, for each sentence we:

\begin{enumerate}

\item[$\bullet$] Identify the center word $u_o$
\item[$\bullet$] Compute the average of the context words $\hat{v}_c = \frac{1}{2w-1} \sum_{-w+o \leq i \leq w+o, i \neq o} v_i$ given window size $w$
\item[$\bullet$] Draw $k$ negative samples $u_1, u_2, .., u_k$
\item[$\bullet$] Compute loss $J(u_o, \hat{v}_c) = -log(\sigma(u_o^T\hat{v}_c)) - \sum_{i=1}^{k} log(\sigma(-u_i^T\hat{v}_c))$
\item[$\bullet$] Update center word vector $u_o$ with gradient $\frac{\partial J(u_o, \hat{v}_c)}{\partial u_o}$
\item[$\bullet$] Update negative word vector $u_i$ with gradient $\frac{\partial J(u_o, \hat{v}_c)}{\partial u_i}$
\item[$\bullet$] Update context word vector $v_i$ with gradient $\frac{\partial J(u_o, \hat{v}_c)}{\partial v_i}$
\end{enumerate}

Center vectors $u_i$ and context vectors $v_j$ are stored full
precision. The final word vectors are the sums of the context and
center vectors $u_i + v_i$ for each corresponding word. The resulting
vectors are full precision.

\subsection{Word2Bits Approach}
To learn quantized word vectors we introduce virtual quantization into the CBOW loss function:
$$
J_{quantized}(u^{(q)}_o, \hat{v}^{(q)}_c) = -log(\sigma((u^{(q)}_{o})^{T} \hat{v}^{(q)}_c)) - \sum_{i=1}^{k} log(\sigma((-u^{(q)}_i)^T\hat{v}^{(q)}_c))
$$

where

$$
u^{(q)}_o = Q_{bitlevel}(u_o)
$$

$$
\hat{v}^{(q)}_c = \sum_{-w+o \leq i \leq w+o,i \neq o} Q_{bitlevel}(v_i)
$$

$$
Q_{bitlevel}(x) = \mbox{ quantization function to quantize to } bitlevel \mbox{ bits}
$$

The following quantization functions were used (chosen based on what worked best)

\[
Q_1(x) =
\begin{cases}
  \frac{1}{3} & x \geq 0\\
  -\frac{1}{3} & x < 0
\end{cases}
\]


\[
Q_2(x) =
\begin{cases}
  \frac{3}{4} & x > \frac{1}{2}\\
  \frac{1}{4} & 0 \leq x \leq \frac{1}{2}\\
  -\frac{1}{4} & -\frac{1}{2} \leq x < 0\\
  -\frac{3}{4} & x < -\frac{1}{2}
\end{cases}
\]

Since $Q_{bitlevel}$ is a discrete function, its derivative is
undefined at some points and 0 at others. To solve this we simply set
the derivative of $Q_{bitlevel}$ to be the identity function:

$$
\frac{\partial Q_{bitlevel}(x)}{\partial x} = x
$$

This is also known as Hinton's straight-through estimator[1].

Like in the standard algorithm, we optimize $J_{quantized}$ with
respect to $u_i$ and $v_j$ over a corpus of text. The final vector
for each word is $Q_{bitlevel}(u_i + v_i)$; thus each parameter takes
$bitlevel$ bits to represent.

Intuitively, although we are still updating $u_i$ and $v_j$ (full
precision vectors), we are now optimizing their quantized
counterparts $Q_{bitlevel}(u_i)$ and $Q_{bitlevel}(v_j)$ to capture
the same corpus statistics as regular word vectors. While we are still
training with full precision 32-bit arithmetic operations and 32-bit floating
point values, the final word vectors we save to disk are
quantized. The full process is presented below.

\begin{algorithm}
\caption{Word2Bits}\label{euclid}
\begin{algorithmic}[1]
\Procedure{MyProcedure}{}
\State $\textit{stringlen} \gets \text{length of }\textit{string}$
\State $i \gets \textit{patlen}$
\BState \emph{top}:
\If {$i > \textit{stringlen}$} \Return false
\EndIf
\State $j \gets \textit{patlen}$
\BState \emph{loop}:
\If {$\textit{string}(i) = \textit{path}(j)$}
\State $j \gets j-1$.
\If {$\textit{string}(i) = \textit{path}(j)$}
\State what
\EndIf
\State $i \gets i-1$.
\State \textbf{goto} \emph{loop}.
\State \textbf{close};
\EndIf
\State $i \gets i+\max(\textit{delta}_1(\textit{string}(i)),\textit{delta}_2(j))$.
\State \textbf{goto} \emph{top}.
\EndProcedure
\end{algorithmic}
\end{algorithm}

\section{Experiments and Results}

\section{Discussion and Future Work}

\subsubsection*{Acknowledgments}

\subsubsection*{References}

\small{ [1] Alexander, J.A. \& Mozer, M.C. (1995) Template-based
  algorithms for connectionist rule extraction. In G. Tesauro,
  D. S. Touretzky and T.K. Leen (eds.), {\it Advances in Neural
    Information Processing Systems 7}, pp. 609-616. Cambridge, MA: MIT
  Press.

- COMPRESSING WORD EMBEDDINGS VIA DEEP COMPOSITIONAL CODE LEARNING
- FASTTEXT.ZIP: COMPRESSING TEXT CLASSIFICATION MODELS
- FASTTEXT (Bag of Tricks for Efficient Text Classification)
- Intrinsic Evaluation of Word Vectors Fails to Predict Extrinsic Performance
- Improving Distributional Similarity with Lessons Learned from Word Embeddings
- Optimal brain damage
- Second order derivatives for network pruning: Optimal brain surgeon
- Learning both weights and connections for efficient neural networks
- Distributed Representations of Words and Phrases and their Compositionality
- Efficient Estimation of Word Representations in Vector Space
- Binarized Neural Networks
- Hinton, Geoffrey. Neural networks for machine learning. Coursera, video lectures, 2012.
- GloVe: Global Vectors for Word Representation
- Learned in Translation: Contextualized Word Vectors
- HALP
- Towards Lower Bounds on Number of Dimensions for Word Embeddings
- Evaluation methods for unsupervised word embeddings
- SQuAD
- DrQA
- WordSim353 Finkelstein
- WordSim Similarity, WordSim Relatedness (Zesch et al., 2008; Agirre et al., 2009
- Bruni et al.s (2012) MEN
- Radinsky et al.s (2011) Mechanical Turk
- Luong et al.s (2013) Rare Words dataset
- Hill et al.s (2014) SimLex-999
- Low precision arithmetic for deep learning.
- Predicting parameters in deep learning
- Compressing neural networks with the hashing trick
- Compression of Neural Machine Translation Models via Pruning
\end{document}
